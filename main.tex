\documentclass[11pt]{article}

\usepackage{hyperref}
\usepackage{xcolor}
\usepackage{calc}
\usepackage{graphicx}
\usepackage{tikz}
\usepackage{fontspec}
\usepackage{fontawesome5}
\usepackage{titlesec}
\usepackage{enumitem}
\usepackage{fancybox}

\hypersetup{hidelinks}

%%%%%%%%%%%%%%%%%%%%
% 设置
%%%%%%%%%%%%%%%%%%%%

\setlength{\parindent}{0pt}
\pagenumbering{gobble}
\setlist[itemize]{nosep
    , before={\vspace*{-\parskip}}
    , leftmargin=*}
\setlist[enumerate]{leftmargin=*}
\renewcommand{\arraystretch}{1.2}
\linespread{1.25}

\titleformat{\section}
  {\LARGE\bfseries\raggedright}
  {}{0em}
  {}
  [{\color{secondary_color}\titlerule}]
\titlespacing*{\section}{0cm}{*1.2}{*1.2}

\usepackage[
	a4paper,
	left=1.2cm,
	right=1.2cm,
	top=1.5cm,
	bottom=1cm,
	nohead
]{geometry}

\setmainfont[
    Path=fonts/,
    Extension=.otf,
    BoldFont=*-Bold,
]{NotoSerifSC}

\definecolor{primary_color}{RGB}{74,164,132}
\definecolor{secondary_color}{RGB}{74,164,132}

\newlength{\iconwidth}
\setlength{\iconwidth}{1.5em}

%%%%%%%%%%%%%%%%%%%%
% 文章内容
%%%%%%%%%%%%%%%%%%%%

\newcommand{\school}{地理科学学院 | School of Geography}

\newcommand{\contact}{
    \small
    \textcolor{black}{
        \faEnvelope \quad \href{mailto:youremail@njnu.edu.cn}{youremail@njnu.edu.cn}
        \hspace{4em}
        \faPhone \quad  130-xxxx-xxxx
        \hspace{4em}
        \faGithub \quad \href{https://github.com/Reborn14}{GitHub 项目地址}
    }
}

\begin{document}

    %%%%%%%%%%%%%%%%%%%%
    % 页眉、页脚和背景
    %%%%%%%%%%%%%%%%%%%%

    \begin{tikzpicture}[remember picture, overlay]
        \node[anchor=north, inner sep=0pt](header) at (current page.north){
            \includegraphics[width=\paperwidth]{images/header.png}
        };
        \node[anchor=west](school_logo) at (header.west){
            \hspace{0.5cm}
            \includegraphics[width=0.3\textwidth]{images/header_nnu.png}
        };
        \node[anchor=east](school_name) at(header.east){
            \textcolor{white}{\textbf{\school}}
            \hspace{0.5cm}
        };
    \end{tikzpicture}
    \vspace{-4.5em}

    \begin{tikzpicture}[remember picture, overlay]
        \node[anchor=south, inner sep=0pt](footer) at (current page.south){
            \includegraphics[width=1.1\paperwidth, height=1cm]{images/foot.png}
        };
        \node[anchor=center] at(footer.center){\contact};
    \end{tikzpicture}

    \begin{tikzpicture}[remember picture, overlay]
        \node[opacity=0.05] at(current page.center){
            \includegraphics[width=0.7\paperwidth, keepaspectratio]{images/nnu_logo.png}
        };
    \end{tikzpicture}

    %%%%%%%%%%%%%%%%%%%%
    % 简历正文
    %%%%%%%%%%%%%%%%%%%%

    \begin{minipage}[t]{0.78\textwidth}
        \begin{minipage}[t]{\textwidth}
        \section[个人信息]{\makebox[\iconwidth][c]{\color{primary_color}{\faAddressCard}}\quad 个人信息}
        \begin{minipage}[t]{0.5\textwidth}
            \textbf{姓\qquad 名}:你的名字

            \vspace{0.5em}
            \textbf{出生年月}:你的出生年月
        \end{minipage}
        \begin{minipage}[t]{0.35\textwidth}
            \textbf{性\qquad 别}:你的性别

            \vspace{0.5em}
            \textbf{政治面貌}:你的政治面貌
        \end{minipage}
        \vspace{1.2em}
        \end{minipage}

        \begin{minipage}[t]{\textwidth}
        \section[教育背景]{\makebox[\iconwidth][c]{\color{primary_color}{\faGraduationCap}}\quad 教育背景}

        {\large \textbf{南京师范大学}},专科 \hfill 2000年9月--2010年6月
        \begin{itemize}
            \item 你的学院,你的专业
            \item \textbf{主修课程}:课程1、课程2、课程3、课程4\ 等。
        \end{itemize}

        \vspace{0.5em}
        {\large \textbf{南京师范大学}},本科 \hfill 2010年9月--2020年6月
        \begin{itemize}
            \item 你的学院,你的专业
            \item \textbf{主修课程}:课程1、课程2、课程3、课程4\ 等。
            \item \textbf{GPA}:4.8 / 4.8(排名:1 / 250)
        \end{itemize}

        \vspace{0.5em}
        {\large \textbf{南京师范大学}},硕士 \hfill 2020年9月--至今
        \begin{itemize}
            \item 你的学院,你的专业,你的导师姓名\ 职称
            \item \textbf{研究方向}:方向1、方向2、方向3、方向4\ 等。
        \end{itemize}

        \vspace{1.2em}
        \end{minipage}
    \end{minipage}
    \hfill
    \begin{minipage}[t]{0.2\textwidth}
        \vspace{2em}
        \setlength{\fboxsep}{0pt}
        \doublebox{\includegraphics[width=\linewidth]{images/avatar.png}}
    \end{minipage}

    \begin{minipage}[t]{\textwidth}
    \section[科研成果]{\makebox[\iconwidth][c]{\color{primary_color}{\faAtom}}\quad 科研成果}

    This is One of Your Paper Published in Conference A.
    \begin{itemize}
        \item \textbf{Mingzi Nide}, Daoshi Nide. \hfill 发表于 \textbf{Conference A}(CCF-A类会议)
        \item 用一句话描述这份论文干了什么\dots\dots
    \end{itemize}

    \vspace{0.5em}
    《另一份论文的标题》
    \begin{itemize}
        \item  \textbf{你的名字}、你的师兄、你的导师 \hfill 发表于 \textbf{某篇期刊} (SCI-1区)
        \item 用一句话描述这份论文干了什么\dots\dots
    \end{itemize}

    \vspace{1.2em}
    \end{minipage}

    \begin{minipage}[t]{\textwidth}
    \section[项目与教学]{\makebox[\iconwidth][c]{\color{primary_color}{\faChalkboardTeacher}}\quad 项目与教学}

    {\large \textbf{项目名称}} \hfill 2020年9月--2021年9月
    \begin{itemize}
        \item \textbf{你在项目中扮演的角色} \hfill 横向/纵向项目-已完结/进行中
        \item 用一句话介绍你在这个项目中做了什么\dots\dots
    \end{itemize}

    \vspace{0.5em}
    {\large \textbf{某某主题讨论班}},主讲 / 参与 \hfill 2020年夏季
    \begin{itemize}
        \item \textbf{主要内容}:内容1,内容2,内容3\ 等。
    \end{itemize}

    \vspace{0.5em}
    {\large \textbf{课程名称}},助教 \hfill 2021年夏季
    \begin{itemize}
        \item \textbf{主要内容}:内容1,内容2,内容3\ 等。
    \end{itemize}

    \vspace{1.2em}
    \end{minipage}

    \begin{minipage}[t]{0.6\textwidth}
        \section[技能特长]{\makebox[\iconwidth][c]{\color{primary_color}{\faWrench}}\quad 技能特长}
        \begin{itemize}
        \setlength{\itemsep}{0.5em}
            \item 熟练使用 Python、Jvav、Rust 等编程语言。
            \item 熟练使用 Tensorflow、Pytorch 等深度学习框架。
            \item 熟悉 Windows 与 Linux 端开发。
        \end{itemize}
    \end{minipage}
    \hfill
    \begin{minipage}[t]{0.35\textwidth}
        \section[兴趣爱好]{\makebox[\iconwidth][c]{\color{primary_color}{\faStar}}\quad 兴趣爱好}
        \begin{itemize}
        \setlength{\itemsep}{0.5em}
            \item 爱好1
            \item 爱好2
            \item 爱好3
            \item 爱好4
        \end{itemize}
    \end{minipage}

    % \newpage
    % % 如有需要,可以添加额外的页面。不要忘记添加页眉页脚和背景相关的代码。

    % % 竞赛经历
    % \section{\makebox[\widthof{\faTrophy}][c]{\color{primary_color}{\faTrophy}}\quad 竞赛经历}
    % \begin{table}[h!]
    %     \begin{tabularx}{\textwidth}{Xp{\widthof{第零负责人}}p{\widthof{国家级-第100名}}p{\widthof{2030年13月}}}
    %         \textbf{比赛1} & 第一负责人 & 国家级-第10名 & 2023年4月 \\
    %         \textbf{比赛2} & 个人参赛 & 国家级-一等奖 & 2023年8月\\
    %         \textbf{比赛3} & 个人参赛 & 省级-一等奖 & 2022年12月\\
    %         % 同理,可以自己加
    %     \end{tabularx}
    % \end{table}

    % % 技能特长
    % \section{\makebox[\widthof{\faWrench}][c]{\color{primary_color}{\faWrench}}\quad 技能特长}
    % \begin{itemize}
    %     \item 熟练使用\Cpp 、Python、Matlab编程语言。
    %     \item 熟悉Windows与Linux端开发。
    %     \item 熟练使用Tensorflow,Pytorch等深度学习框架。
    %     \item 熟练掌握\Cpp 与Python环境下OpenCV与Qt应用的开发,且熟练使用Qt Creator软件。
    %     \item 熟练使用Altium Designer与LCEDA进行封装绘制与板子设计。
    %     \item 熟练使用Keil,Arduino IDE等集成开发软件。
    %     \item 了解模式识别,强化学习,遗传算法,知识蒸馏等相关概念。
    % \end{itemize}

    % % 所获荣誉
    % \section{\makebox[\widthof{\faStar}][c]{\color{primary_color}{\faStar}}\quad 所获荣誉}
    % \begin{multicols}{2}
    %     \begin{itemize}
    %         \item 某年学业先进个人
    %         \item 某年某奖学金某等奖
    %         \item 某大使
    %         \item 某年某奖学金某等奖
    %         \item 某年优秀团员称号
    %         \item 某年某称号
    %     \end{itemize}
    % \end{multicols}

    % % 其他
    % \section{\makebox[\widthof{\faInfo}][c]{\color{primary_color}{\faInfo}}\quad 其他}
    % \begin{itemize}
    %     \item 英语水平-CET6级xxx分
    %     \item 计算机几级证书
    %     \item xx几级证书
    %     \item 技术博客: 某网址
    %     \item 教师资格证:xxx
    %     \item 普通话证书:几级几等
    %     \item 文字排版:\LaTeX
    % \end{itemize}

\end{document}